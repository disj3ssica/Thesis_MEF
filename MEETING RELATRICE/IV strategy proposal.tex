\documentclass{article}
\begin{document}

    \section{IV proposal}
    $Y$ can be:
    \begin{enumerate}
        \item employment status (can only be done on a smaller portion of the dataset, pre-retirement: 50-66ish y.o.)
        \item social network quality and quantity (convincing?)
            \begin{enumerate}
                \item in this case having a regressor pandemic=1 or similar is crucial to avoid conflating the effects of pandemic and MH worsening 
            \end{enumerate}
        \item health outcomes
            \begin{enumerate}
                \item decline in cognitive ability + memory tasks performance
                \item self-reliance in everyday activities
                \item physical activity 
                \item grip strenght (not sure what it could measure)
                \item needing external care (from family and professionals)
                \item prevalence/worsening of medical conditions (heart disease, dementia\dots)
            \end{enumerate}
    \end{enumerate}
Identification issues:
\begin{enumerate}
    \item Reverse causality for all possible Y 
    \item Measurement error in MH (measured through questionnaires and self-reports)
    \item ???
\end{enumerate}
IV strategy:
\begin{enumerate}
    \item Instruments often used in the literature:
        \begin{enumerate}
            \item number of psychiatric disorders before 18 (available in SHARE? to check)
            \item religious attendance to handle problems (used in Chatterji et al 2007, not convincing)
            \item number of parents' psychiatric disorders 
            \item recent friends' death (not family to avoid correlation with model error, very good)
        \end{enumerate}

    \item My other instruments proposals:
        \begin{enumerate}
            \item Number of days spent under strict lockdown measures:
                \begin{enumerate}
                    \item DATASET: Oxford COVID19 Government Response Tracker (n of deaths per 100000 and n of days with stringent control measures). 
                    \begin{enumerate}
                        \item BEST: OxCGRT daily STRINGENCY INDEX. \newline Link: https://ourworldindata.org/covid-stringency-index . Filename: covid-containment-and-health-index .
                        \item Stringency Index metrics: school closures; workplace closures; cancellation of public events; restrictions on public gatherings; closures of public transport; stay-at-home requirements; public information campaigns; restrictions on internal movements; and international travel controls.
                        \item Containment and Health Index metrics: school closures; workplace closures; cancellation of public events; restrictions on public gatherings; closures of public transport; stay-at-home requirements; public information campaigns; restrictions on internal movements; international travel controls; testing policy; extent of contact tracing; face coverings; and vaccine policy. BETTER ONE.
                        \item Both indexes between 0 and 100 (strictest).
                    \end{enumerate}
                    I could take the past 30 days average of strictness of covid measures. 
                    \item Travel restrictions implemented to control the spread of COVID-19 are lifted in the EU. \newline Link: https://reopen.europa.eu/en . 
                \end{enumerate}
            \item Number of deaths. OxCGRT tracker data. Include STRICTNESS + DEATHS because strictness itself may reflect country specific risk aversion but not actual impact. Both impact MH.
            \item Severity of childhood conditions (abuse, neglect - should check the detail of this info in SHARE)
            \item recent death in social network
        \end{enumerate}
\end{enumerate}
Covariates in the model: 
\begin{enumerate}
    \item Individual characteristics (age, gender, educ, living conditions, income\dots)
    \item Country or region FE
    \item Having vaccine available or not (OxGRT data)
    \item Others: \dots
    \item Age, gender, marital status (dn014\_), spouse alive, spose in good health, divorced (dn018\_). Check coverscreen for all of these and more.
    \item Number of children
    \item Number of grandchildren
    \item Number of siblings
    \item Amount spent on food at home (co002e) and amount spent on food outside the home (co003e) as proxy for going out.
    \item Is household able to make ends meet (co007\_) for financial stress.
    \item Consume home produced food (co010\_).
    \item Household size
    \item Immigrant (dn004\_)
    \item Education level (dn010\_) and years of educ (dn041\_)
    \item Natural parent still alive (dn026\_1 mother and dn026\_2 father)
    \item Health of parent (dn033\_1 mother, and dn033\_2 father)
    \item Employment and pensions (ep coded vars)
    \item adl variable measures limitations with daily activities
    \item 
\end{enumerate}
Predicting MH:
\begin{enumerate}
    \item EURO-D depression scale
    \item UCLA loneliness scale
    \item SN size and quality
    \item "mh" coded vars 
    \item "te" vars for time expenditure information 
    \item conversation vars (from ch\_q10a to ch\_q10e)
    \item 
\end{enumerate}
Notes:
\begin{enumerate}
    \item (firstwave) variable identifies first time individual appeared in SHARE! Use to find longitudinal sample.
    \item "ex" coded vars measure EXPECTATIONS about the future, good for MH prediction?
    \item "cv\_" coded vars give better named vars on children information
    \item NUTS level 1 information on individuals location (REGIONAL, very good).
    \item SOCIAL NETWORK has been tracked across waves 6 - 7 - 8, meaning we know how and if the networks changed. How can I use this information?
\end{enumerate}

\end{document}
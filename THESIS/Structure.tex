\documentclass{article}

\begin{document}
\section{Thesis structure}

\begin{enumerate}
    \item ABSTRACT

    \item INTRODUCTION

    \item THE ISSUES OF MENTAL HEALTH 
    \begin{enumerate}
        \item Why is MH relevant
        \begin{enumerate}
            \item A rising problem (find statistics from many different countries)
            \begin{enumerate}
                \item MH issues: depression, anxiety, ADHD, OCD, BPD\dots
                \item DSM-V definitions of disorders
                \item Which disorders do I focus on?
            \end{enumerate}
            \item MH and social capital
            \item MH and public spending (prevention and treatment)
            \item MH and private spending
            \item MH and healthcare spending (+talk about comorbidity between MH issues )
            \item MH and private choice (savings, childbearing, employment)
        \end{enumerate}
        \item Provide literature on MH in general (brief history, up to date literature) + on specific issues brought up in the previous point
        \item How is it measured? 
        \begin{enumerate}
            \item Psychometric tools (most common scales, have they changed during the years)
            \item Reliability of questionnaires (self-administered and under supervision administered)
            \item Most frequently used ones + have they changed recently? How do their scores behave?
        \end{enumerate}
    \end{enumerate}

    \item FRAMING THE RESEARCH QUESTION
    \begin{enumerate}
        \item Q: how can we study the effect of MH on Y (education, free time use, employment, social networks quality and quantity, health outcomes)?
        \item Literature review focused on this question: strategies, weaknesses of most common strategies.
        \item Which data would I have in an ideal world?
        \item Which analysis would I do in the ideal setting?
        \item Using the pandemic as an identification tool. 
        \item The identification challenge. 
    \end{enumerate}

    \item THE DATA
    \begin{enumerate}
        \item SHARELIFE DATASET 
        \begin{enumerate}
        \item Describe the dataset.
        \item What variables are relevant to me?
        \item Building a MH indicator + reference other literature.
        \item Plots and other graphic representation to explore and understand the dataset. 
        \item Potential collinearity between variables. 
        \end{enumerate}
        \item PANDEMIC RESTRICTIVENESS DATA
        \begin{enumerate}
            \item Describe data 
        \end{enumerate}
    \end{enumerate}
        

    \item IDENTIFICATION STRATEGY
    \begin{enumerate}
        \item IV approach
        \begin{enumerate}
            \item The method (brief explanation)
            \item The instruments + novelty of my new instrument
            \item Weaknesses + strenghts
            \item How to evaluate method performance
        \end{enumerate}
        \item ML/non-parametric approach to the first stage of IV
        \begin{enumerate}
            \item The method
            \item Advantages and disadvantages with respect to regular IV
            \item How to evaluate method performance
        \end{enumerate} 
    \end{enumerate}

    \item ESTIMATION
    \begin{enumerate}
        \item Data cleaning (variables transformation)
        \item Run the models
        \item Plot results
        \item Comment results
    \end{enumerate}

    \item DISCUSSION
    \begin{enumerate}
        \item What did I find?
        \item How does it compare with the most reliable literature?
        \item How do the IV and IV+ML results compare?
    \end{enumerate}

    \item CONCLUSION
    \begin{enumerate}
        \item Recap
        \item How could the work be improved?
        \item What is the main contribution of my work?
    \end{enumerate}

\end{enumerate}

\end{document}
\chapter{References}

\begin{sortedlist}
    \sortitem{}
    \sortitem{Santini, Ziggi Ivan, et al. ``The moderating role of social network size in the temporal association between formal social participation and mental health: a longitudinal analysis using two consecutive waves of the Survey of Health, Ageing and Retirement in Europe (SHARE)." Social Psychiatry and Psychiatric Epidemiology 56 (2021-B): 417-428.}
    \sortitem{Rönnblad, Torkel, et al. ``Precarious employment and mental health." Scandinavian journal of work, environment and health 45.5 (2019): 429-443.}
    \sortitem{Nadinloyi, Karim Babayi, Hasan Sadeghi, and Nader Hajloo. ``Relationship between job satisfaction and employees mental health." Procedia-Social and Behavioral Sciences 84 (2013): 293-297.}
    \sortitem{Coleman, Max E., et al. ``What kinds of social networks protect older adults’ health during a pandemic? The tradeoff between preventing infection and promoting mental health." Social Networks 70 (2022): 393-402.}
    \sortitem{Adams-Prassl, Abi, et al. ``The impact of the coronavirus lockdown on mental health: evidence from the United States." Economic Policy 37.109 (2022): 139-155.}
    \sortitem{Gu, Naeun. ``The effects of neighborhood social ties and networks on mental health and well-being: A qualitative case study of women residents in a middle-class Korean urban neighborhood." Social Science and Medicine 265 (2020): 113336.}
    \sortitem{Cohen-Cline, Hannah, et al. ``Associations between social capital and depression: A study of adult twins." Health and place 50 (2018): 162-167.}
    \sortitem{Riumallo-Herl, Carlos Javier, Ichiro Kawachi, and Mauricio Avendano. ``Social capital, mental health and biomarkers in Chile: Assessing the effects of social capital in a middle-income country." Social science and medicine 105 (2014): 47-58.}
    \sortitem{Landstedt, Evelina, et al. ``Disentangling the directions of associations between structural social capital and mental health: Longitudinal analyses of gender, civic engagement and depressive symptoms." Social Science and Medicine 163 (2016): 135-143.}
    \sortitem{Ehsan, Annahita M., and Mary J. De Silva. ``Social capital and common mental disorder: a systematic review." J Epidemiol Community Health 69.10 (2015): 1021-1028.}
    \sortitem{Murayama, Hiroshi, et al. ``Do bonding and bridging social capital affect self-rated health, depressive mood and cognitive decline in older Japanese? A prospective cohort study." Social Science and Medicine 98 (2013): 247-252.}
    \sortitem{Sirven, Nicolas, and Thierry Debrand. ``Social capital and health of older Europeans: Causal pathways and health inequalities." Social Science and Medicine 75.7 (2012): 1288-1295.}
    \sortitem{Shiovitz-Ezra, Sharon, and Sara A. Leitsch. ``The role of social relationships in predicting loneliness: The national social life, health, and aging project." Social Work Research 34.3 (2010): 157-167.}
    \sortitem{Ferreira-Alves, José, et al. ``Loneliness in middle and old age: Demographics, perceived health, and social satisfaction as predictors." Archives of gerontology and geriatrics 59.3 (2014): 613-623.}
    \sortitem{Hajek, André, and Hans-Helmut König. ``Which factors contribute to loneliness among older Europeans? Findings from the survey of health, ageing and retirement in Europe: determinants of loneliness." Archives of gerontology and geriatrics 89 (2020): 104080.}
    \sortitem{Santini, Ziggi Ivan, and Ai Koyanagi. ``Loneliness and its association with depressed mood, anxiety symptoms, and sleep problems in Europe during the COVID-19 pandemic." Acta neuropsychiatrica 33.3 (2021-A): 160-163.}
    \sortitem{Jarach, Carlotta Micaela, et al. ``Social isolation and loneliness as related to progression and reversion of frailty in the Survey of Health Aging Retirement in Europe (SHARE)." Age and ageing 50.1 (2021): 258-262.}
    \sortitem{Sunwoo, Lee. ``Loneliness among older adults in the Czech Republic: A socio-demographic, health, and psychosocial profile." Archives of Gerontology and Geriatrics 90 (2020): 104068.}
    \sortitem{Luchetti, Martina, et al. ``Loneliness is associated with risk of cognitive impairment in the Survey of Health, Ageing and Retirement in Europe." International journal of geriatric psychiatry 35.7 (2020): 794-801.}
    \sortitem{Niedzwiedz, Claire L., et al. ``The relationship between wealth and loneliness among older people across Europe: Is social participation protective?." Preventive medicine 91 (2016): 24-31.}
    \sortitem{Fokkema, Tineke, Jenny De Jong Gierveld, and Pearl A. Dykstra. ``Cross-national differences in older adult loneliness." The Journal of psychology 146.1-2 (2012): 201-228.}
    \sortitem{Santini, Ziggi Ivan, et al. ``Social disconnectedness, perceived isolation, and symptoms of depression and anxiety among older Americans (NSHAP): a longitudinal mediation analysis." The Lancet Public Health 5.1 (2020): e62-e70.}
    \sortitem{Luo, Ye, et al. ``Loneliness, health, and mortality in old age: A national longitudinal study." Social science and medicine 74.6 (2012): 907-914.}
    \sortitem{Arpino, Bruno, et al. ``Loneliness before and during the COVID-19 pandemic—are unpartnered and childless older adults at higher risk?." European journal of ageing 19.4 (2022): 1327-1338.}
    \sortitem{Atzendorf, Josefine, and Stefan Gruber. ``Depression and loneliness of older adults in Europe and Israel after the first wave of covid-19." European journal of ageing 19.4 (2022): 849-861.}
    \sortitem{Frijters, Paul, David W. Johnston, and Michael A. Shields. ``Mental health and labour market participation: Evidence from IV panel data models." (2010).}
    \sortitem{Banerjee, Souvik, Pinka Chatterji, and Kajal Lahiri. ``Effects of psychiatric disorders on labor market outcomes: a latent variable approach using multiple clinical indicators." Health economics 26.2 (2017): 184-205.}
    \sortitem{Lakhan, Ram, Amit Agrawal, and Manoj Sharma. ``Prevalence of depression, anxiety, and stress during COVID-19 pandemic." Journal of neurosciences in rural practice 11.04 (2020): 519-525.}
    \sortitem{Wang, Cuiyan, et al. ``A longitudinal study on the mental health of general population during the COVID-19 epidemic in China." Brain, behavior, and immunity 87 (2020): 40-48.}
    \sortitem{Deng, Jiawen, et al. ``The prevalence of depression, anxiety, and sleep disturbances in COVID‐19 patients: a meta‐analysis." Annals of the New York Academy of Sciences 1486.1 (2021): 90-111.}
    \sortitem{Pieh, Christoph, et al. ``Mental health during COVID-19 lockdown in the United Kingdom." Psychosomatic medicine 83.4 (2021): 328-337.}
    \sortitem{Wolitzky‐Taylor, Kate B., et al. ``Anxiety disorders in older adults: a comprehensive review." Depression and anxiety 27.2 (2010): 190-211.}
    \sortitem{Schaub, Rainer T., and Michael Linden. ``Anxiety and anxiety disorders in the old and very old—results from the Berlin Aging Study (BASE)." Comprehensive psychiatry 41.2 (2000): 48-54.}
    \sortitem{Morrow-Howell, Nancy, et al. ``Depression in public community long-term care: implications for intervention development." The Journal of Behavioral Health Services and Research 35 (2008): 37-51.}
    \sortitem{Zenebe, Yosef, Baye Akele, and Mogesie Necho. ``Prevalence and determinants of depression among old age: a systematic review and meta-analysis." Annals of general psychiatry 20.1 (2021): 1-19.}
    \sortitem{OECD/EU, ``Health at a Glance: Europe 2018: State of Health in the EU Cycle", OECD Publishing, Paris, (2018).  \url{https://doi.org/10.1787/health_glance_eur-2018-en}.}
    \sortitem{OECD/EU, ``Health at a Glance: Europe 2022: State of Health in the EU Cycle", OECD Publishing, Paris, (2022). \url{https://doi.org/10.1787/507433b0-en}.}
    \sortitem{World Health Organization, ``World mental health report: transforming mental health for all", Geneva, 2022. Licence: CC BY-NC-SA 3.0 IGO.}
    \sortitem{Lakner, Christoph, et al. ``How much does reducing inequality matter for global poverty?." The Journal of Economic Inequality 20.3 (2022): 559-585.}
    \sortitem{Global Burden of Disease Collaborative Network. ``Global Burden of Disease Study 2019 (GBD 2019) Results". Seattle, United States: Institute for Health Metrics and Evaluation (IHME), (2020). Available from \url{https://vizhub.healthdata.org/gbd-results/}.}
    \sortitem{Woody, C. A., et al. ``A systematic review and meta-regression of the prevalence and incidence of perinatal depression." Journal of affective disorders 219 (2017): 86-92.}
    \sortitem{Alexandrino-Silva, Clóvis, et al. ``Gender differences in symptomatic profiles of depression: results from the Sao Paulo Megacity Mental Health Survey." Journal of affective disorders 147.1-3 (2013): 355-364.}
    \sortitem{Kessler, Ronald C., et al. ``Lifetime prevalence and age-of-onset distributions of DSM-IV disorders in the National Comorbidity Survey Replication." Archives of general psychiatry 62.6 (2005): 593-602.}
    \sortitem{Chesney, Edward, Guy M. Goodwin, and Seena Fazel. ``Risks of all‐cause and suicide mortality in mental disorders: a meta‐review." World psychiatry 13.2 (2014): 153-160.}
    \sortitem{Vieira, Edgar Ramos, Ellen Brown, and Patrick Raue. ``Depression in older adults: screening and referral." Journal of geriatric physical therapy 37.1 (2014): 24-30.}
\end{sortedlist}
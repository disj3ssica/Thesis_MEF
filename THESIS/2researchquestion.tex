\chapter{Framing the Research Question}

Building upon the previous chapter's exploration of mental health, this chapter aims to frame the research question and provide a comprehensive and pertinent literature review. While there exists an extensive body of literature on mental health, both as an isolated subject and as a determinant of individual outcomes, much of this research is limited to correlational analyses. These studies often fall short of addressing the methodological challenges inherent in establishing a causal relationship between mental health conditions and individual outcomes.

Among the most commonly examined outcomes related to mental health are:
\begin{enumerate}
    \item \textbf{Labor Market Participation.} Including employment status, job performance, hours worked, wages, and self-rated satisfaction.
    \item \textbf{Physical Well-being.} Such as likelihood of hospitalization, quality of life, healthcare utilization, physical mobility, and dependence on external assistance.
    \item \textbf{Social Networks.} Measured by size and quality, self-reported satisfaction, frequency of social interaction, isolation, and perceived loneliness.
    \item \textbf{Community Involvement.} Including civic participation and elective activities.
\end{enumerate}

Additional outcomes may include behavioral ones such as the likelihood of substance use, financial stability, and educational ones like dropout rates, attainment, attendance, and performance.

Data collection for mental well-being is typically conducted using standardized questionnaires and scales, including Beck Depression Inventory (BDI), Generalized Anxiety Disorder 7 (GAD-7), Patient Health Questionnaire 9 (PHQ-9), Center for Epidemiological Studies-Depression Minus Loneliness (CES-D-ML), %  FINISH  % . 
These assessments are commonly administered via assisted face-to-face interviews or through computerized adaptive testing interviews (CATI). Observations made by the interviewer about the context and the respondent can be integrated to provide a more comprehensive understanding of the individual's state.

\section{Literature Review}
%   introduce the literature review
The existing literature on the topic is fragmented in both topics and methods, primarily due to challenges in sourcing appropriate data for investigation and different diagnostic tools employed to assess mental well-being in subjects. 
A frequently utilized dataset for this line of research is the Survey of Health, Ageing and Retirement in Europe (SHARE). This dataset provides a wealth of variables that are highly relevant to this study, thus the following literature review is particularly focused in its applications in researching mental health. Detailed information about SHARE, as well as other datasets employed in this dissertation, will be available in Chapter 3.


%   introduce the methodology for the literature review
    %   Clearly state what you aim to discover through this literature review
    %   Inclusion Criteria
    %   -   Peer-reviewed articles only.
    %   -   Published within a specific time frame (e.g., last 10 years).
    %   -   Articles must focus on empirical research.
    %   -   Articles must be available in English


\subsection{Mental Health and Labor Market Outcomes}


\subsection{Mental Health and Social Capital}


\subsection{Mental Health and Social Networks}


\subsection{Mental Health and Loneliness}


\subsection{GAPS IN THE LITERATURE: Comparative Analysis}


\subsection{Conclusion}



%The overarching question guiding this study is: "How does mental health, as measured by various indicators, affect individual outcomes such as labor market participation, community activity, and social network strength?" To answer this question, we employ a dataset known as the Survey of Health, Ageing and Retirement in Europe (SHARE), which offers a rich array of variables pertinent to our study.



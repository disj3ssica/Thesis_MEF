\chapter{Framing the Research Question}

Building upon the previous chapter's exploration of mental health, this chapter aims to frame the research question and provide a comprehensive and pertinent literature review. While there exists an extensive body of literature on mental health, both as an isolated subject and as a determinant of individual outcomes, much of this research is limited to correlational analyses. These studies often fall short of addressing the methodological challenges inherent in establishing a causal relationship between mental health conditions and individual outcomes.

Among the most commonly examined outcomes related to mental health are:
\begin{enumerate}
    \item \textbf{Labor Market Participation.} Including employment status, job performance, hours worked, wages, and self-rated satisfaction.
    \item \textbf{Physical Well-being.} Such as likelihood of hospitalization, quality of life, healthcare utilization, physical mobility, and dependence on external assistance.
    \item \textbf{Social Networks.} Measured by size and quality, self-reported satisfaction, frequency of social interaction, isolation, and perceived loneliness.
    \item \textbf{Community Involvement.} Including civic participation and elective activities.
\end{enumerate}

Additional outcomes may include behavioral ones such as the likelihood of substance use, financial stability, and educational ones like dropout rates, attainment, attendance, and performance.

Data collection for mental well-being is typically conducted using standardized questionnaires and scales, including Beck Depression Inventory (BDI), Generalized Anxiety Disorder 7 (GAD-7), Patient Health Questionnaire 9 (PHQ-9), Center for Epidemiological Studies-Depression Minus Loneliness (CES-D-ML), %  FINISH  % . 
These assessments are commonly administered via assisted face-to-face interviews or through computerized adaptive testing interviews (CATI). Observations made by the interviewer about the context and the respondent can be integrated to provide a more comprehensive understanding of the individual's state.


\section{Literature review}
    The existing literature on the topic is fragmented, primarily due to challenges in sourcing appropriate data for investigation and different diagnostic tools employed to assess mental well-being in subjects. 
    A frequently utilized dataset for this line of research is the Survey of Health, Ageing and Retirement in Europe (SHARE). This dataset provides a wealth of variables that are highly relevant to this study, thus the following literature review is particularly focused in its applications in researching mental health. Detailed information about SHARE, as well as other datasets employed in this dissertation, will be available in Chapter 3.

    \subsection{MH and labor market participation}
    %   *** Banerjee et al (2017), Effects Of Psychiatric Disorders On Labor Market Outcomes: A Latent Variable Approach Using Multiple Clinical Indicators 
    %           --> ENDOGENEITY ADDRESSED WITH COVARIANCE INSTRUMENTS + MULTIPLE INDICATOR AND MULTIPLE CAUSE MODEL
    %   Frijters et al. (2010), Mental Health and Labour Market Participation: Evidence from IV Panel Data Models  
    %           --> IV + PANEL


    \subsection{MH and loneliness/isolation}
    %   *** Atzendorf Gruber (2022), Depression and loneliness of older adults in Europe and Israel after the first wave of covid 19 
    %           --> BINARY LOGIT (NOT IV) + SHARE + OXGRT 
    %   *** Arpino et al (2022), Loneliness before and during the COVID 19 pandemic—are unpartnered and childless older adults at higher risk? 
    %           --> LOGIT + COVID + SHARE + POPULATION: UNPARTNERED, CHILDLESS, OLDER
    %   **  Luo et al (2012),  Loneliness, health, and mortality in old age: A national longitudinal study
    %           --> MORTALITY OF LONELINESS + SN + CROSS LAGGED PANEL + AUTOREG + REVERSE CAUSALITY 
    %   **  Santini et al (2020), Social disconnectedness, perceived isolation, and symptoms of depression and anxiety among older Americans (NSHAP): a longitudinal mediation analysis 
    %           --> LONGITUDINAL MEDIATION ANALYSIS + EVIDENCE OF REVERSE CAUSALITY
    %   Fokkema et al. 2012, Cross-national differences in older adult loneliness 
    %           --> LOGISTIC MODEL
    %   Niedzwiedz et al. 2016, The relationship between wealth and loneliness among older people across Europe: Is social participation protective? 
    %           --> LOGISTIC + SHARE DATA W5
    %   Luchetti et al 2019, Loneliness is associated with risk of cognitive impairment in the Survey of Health, Ageing and Retirement in Europe 
    %           --> COX REGRESSION + SHARELIFE DATA 
    %   Lee 2020, Loneliness among older adults in the Czech Republic: A socio-demographic, health, and psychosocial profile 
    %           --> SIMPLE REGRESSION + TOPICS: SN, WELL-BEING
    %   Jarach et al (2021), Social isolation and loneliness as related to progression and reversion of frailty in the Survey of Health Aging Retirement in Europe (SHARE) 
    %           --> LOGISTIC REGRESSION + FRAILTY 
    %   Santini, Koyanagi (2021), Loneliness and its association with depressed mood, anxiety symptoms, and sleep problems in Europe during the COVID-19 pandemic 
    %           --> MULTIVARIATE LOGIT + COVID + SHARE
    %   Hajek, Konig (2022), Which factors contribute to loneliness among older Europeans? Findings from the Survey of Health, Ageing and Retirement in Europe Determinants of loneliness 
    %           --> LINEAR FE + SHARE + PANEL
    %   Alves et al (2014), Loneliness in middle and old age: Demographics, perceived health, and social satisfaction as predictors 
    %           --> PREDICTING LONELINESS + NOT CAUSAL + CORRELATIONS 
    

    \subsection{MH and social networks}
    %   Shiovitz-Ezra, Leitsch (2010), The Role of Social Relationships in Predicting Loneliness: The National Social Life, Health, and Aging Project 
    %           --> MH = LONELINESS WRT MARITAL STATUS + HIERARCHICAL LINEAR REGRESSION

    \subsection{MH and social capital}
    %   *** Sirven Debrand 2012, Social capital and health of older Europeans: Causal pathways and health inequalities 
    %           --> GRANGER CAUSALITY BETWEEN MH AND SOC CAP + SHARELIFE + BIVARIATE PROBIT
    %   Murayama et al (2013), Do bonding and bridging social capital affect self-rated health, depressive mood and cognitive decline in older Japanese? A prospective cohort study 
    %           --> LOGIT
    %   Eshan De Silva (2015), Social capital and common mental disorder: a systematic review 
    %           --> REVIEW OF 39 + SC CAN PREVENT MH ISSUES




%The overarching question guiding this study is: "How does mental health, as measured by various indicators, affect individual outcomes such as labor market participation, community activity, and social network strength?" To answer this question, we employ a dataset known as the Survey of Health, Ageing and Retirement in Europe (SHARE), which offers a rich array of variables pertinent to our study.


% LITERATURE REVIEW VERA E PROPRIA

% where does previous literature fall short?


\chapter{Framing the Research Question}

Building upon the previous chapter's exploration of mental health, this chapter aims to frame the research question and provide a comprehensive and pertinent literature review. While there exists an extensive body of literature on mental health, both as an isolated subject and as a determinant of individual outcomes, much of this research is limited to correlational analyses. These studies often fall short of addressing the methodological challenges inherent in establishing a causal relationship between mental health conditions and individual outcomes.

Among the most commonly examined outcomes related to mental health are:
\begin{enumerate}
    \item \textbf{Labor Market Participation.} Including employment status, job performance, hours worked, wages, and self-rated satisfaction.
    \item \textbf{Physical Well-being.} Such as likelihood of hospitalization, quality of life, healthcare utilization, physical mobility, and dependence on external assistance.
    \item \textbf{Social Networks.} Measured by size and quality, self-reported satisfaction, frequency of social interaction, isolation, and perceived loneliness.
    \item \textbf{Community Involvement.} Including civic participation and elective activities.
\end{enumerate}

Additional outcomes may include behavioral ones such as the likelihood of substance use, financial stability, and educational ones like dropout rates, attainment, attendance, and performance.

Data collection for mental well-being is typically conducted using standardized questionnaires and scales, including Beck Depression Inventory (BDI), Generalized Anxiety Disorder 7 (GAD-7), Patient Health Questionnaire 9 (PHQ-9), Center for Epidemiological Studies-Depression Minus Loneliness (CES-D-ML), %  FINISH  % . 
These assessments are commonly administered via assisted face-to-face interviews or through computerized adaptive testing interviews (CATI). Observations made by the interviewer about the context and the respondent can be integrated to provide a more comprehensive understanding of the individual's state.

\section{Literature Review}
%   introduce the literature review
The existing literature on the topic is fragmented in both topics and methods, primarily due to challenges in sourcing appropriate data for investigation and different diagnostic tools employed to assess mental well-being in subjects. 
A frequently utilized dataset for this line of research is the Survey of Health, Ageing and Retirement in Europe (SHARE). This dataset provides a wealth of variables that are highly relevant to this study, thus the following literature review is particularly focused in its applications in researching mental health. Detailed information about SHARE, as well as other datasets employed in this dissertation, will be available in Chapter 3.


%   introduce the methodology for the literature review
    %   Clearly state what you aim to discover through this literature review
    %   Inclusion Criteria
    %   -   Peer-reviewed articles only.
    %   -   Published within a specific time frame (e.g., last 10 years).
    %   -   Articles must focus on empirical research.
    %   -   Articles must be available in English


\subsection{Mental Health and Labor Market Outcomes}
    The relationship between mental health and labor market outcomes is intricate, which may explain why existing research on the subject is limited and largely focuses on correlational findings. Labor market conditions encompass a wide range of factors, including job security, work-life balance, income levels, self-assessed job satisfaction, social support, and employment status. Additionally, specific working conditions—such as remote versus in-person work—and skill mismatches can influence an individual's mental well-being. In turn, these mental health states can also impact labor market outcomes. A systematic review by Rönnblad et al. (2019) investigated the effects of precarious employment on mental health, and mostly found very low quality evidence of negative effects of temporary employment or unpredictable work hours on mental health, and moderate quality of evidence was found for perceived job insecurity having adverse effects on mental health. 

    As an example of the many correlational studies, in Nadinloyi et al.'s (2013), the authors explore the correlation between job satisfaction and mental health among employees in two industrial firms. They assess individual conditions using Birfield's Job Satisfaction Scale and the Ruth Questionnaire and Scale. The study employs multiple regression analysis, t-tests, and Pearson correlation coefficients as its methodology. However, it does not address the potential issue of reverse causality between job satisfaction and mental well-being, thereby limiting the interpretation of the results to correlational rather than causal relationships. 

    In contrast to the method employed by Nadinloyi et al. (2013), Banerjee et al. (2017), Frijters et al. (2014) and Frijters et al. (2010) tackle endogeneity issues in two different ways. Banerjee et al. (2017) explore the impact of psychiatric disorders on labor market performance by utilizing a structural equation model that incorporates a latent index for mental health. This index is formulated based on symptoms from four specific psychiatric conditions (major depression, panic attacks, social phobia, and generalized anxiety disorder) as well as demographic, socioeconomic, and health-related variables. To address endogeneity, the study employs a Multiple Indicator and Multiple Cause (MIMIC) model, with the aid of covariance instruments. The findings reveal that mental illness negatively influences both employment rates and labor force participation. The study estimates that improving mental health could potentially increase employment for 3.5 million people and reduce absenteeism costs by approximately \$21.6 billion.
    Frijters et al. (2010) focus on the impact of mental health on employment status. Mental health is measured as an index based on the Short-Form General Health Survey (SF-36) answers. To tackle endogeneity concerns, their preferred specification is an Instrumental Variable (IV) Probit model, using the death of a close friends as an instrument for mental health to control for endogeneity concerns. The paper finds that a one standard deviation decline in mental health leads to a drop in the probability of labor market participation by around 19 percentage points. 
    Finally, Frijters et al. (2014) also measures mental well-being with the SF-36 Survey and exploits the death of a close friend as an instrument, however the method of choice is an Instrumental Variables Fixed Effects (IV-FE) model applied to high-quality Australian panel data spanning 10 waves. Results prove that mental health has a substantial negative impact on employment, with a one standard deviation in mental health leading to a 30 percentage point reduction in the likelihood of being employed. 


\subsection{Mental Health and Loneliness}
% difference between isolation and loneliness, and their relation
% relevance for mental health
    The body of literature exploring the relationship between loneliness and mental health faces the same methodological challenges, including issues of reverse causality, unobserved variables, and measurement errors in the independent variable. Studies in this domain can be categorized into three distinct groups: 
    1) pre-pandemic studies that largely fall short in adequately addressing endogeneity concerns; 
    2) research leveraging pandemic-related data to investigate the link between loneliness and mental health;
    3) a subset of papers employing more rigorous methodologies to provide credible insights into the relationship.

    Fokkema et al. (2012) employ a cross-country comparative approach to analyse loneliness among older adults. Health variables include perceived health, functional limitations, and problems with seeing or hearing, all measured on a 5-point scale ranging from 'excellent' to 'poor.' The study utilizes hierarchical logistic regression to explore the factors contributing to varying levels of loneliness across countries. The dependent variable, 'loneliness,' is assessed through a single-item measure derived from the CES-D (depression) scale.
    The findings indicate that countries with older populations, a higher proportion of women, and a greater number of unpartnered older adults tend to report elevated levels of loneliness. However, the unaddressed endogenous relationship between physical and mental health limits the causal interpretation of the results.

    A paper by Alves et al. (2014) aims at understanding the predictors of feelings of loneliness in middle-aged and older adults in Portugal through logistic regression analysis using survey data (socio-demographic variables, residence characteristics, measures of health). They find that variables such as age, gender, marital status, living arrangements, region, type of housing, professional status and income are all significantly associated with feelings of loneliness. 

    Niedzwiedz et al. (2016) investigates the relationship of loneliness and household wealth in older adults, focusing on the mediating role of social participation. Mental well-being is measured with the R-UCLA loneliness scale, and household wealth is measured by the sum of financial and real assets, minus liabilities. The authors recognize the limitations of a cross-sectional logistic regression study, and find that household wealth is associated with higher levels of loneliness. They also identify social participation as a key mediating factor, noting that certain forms of social engagement are particularly effective in alleviating loneliness.

    Logistic regression is also the tool of choice in Jarach et al. (2021), which uses SHARE data to investigate the relation between loneliness and the reversion of frailty in older Europeans. Loneliness is measured with the UCLA-L scale, and social isolation is measured with a custom index. 
    Multinomial logistic regression is used to compute relative risk ratios for changing frailty status according to levels of social isolation and loneliness. Their findings indicate that both loneliness and social isolation are significantly linked to the increased risk of individuals transitioning from a robust to a frail or pre-frail state.

    Loneliness may also have an association with cognitive impairment, as analysed by Luchetti et al. (2019), which investigate the relationship between loneliness and cognitive impairment using data from SHARE. To assess cognitive performance, they utilize the memory and verbal fluency tasks provided by SHARE, while employing the R-UCLA scale to gauge loneliness. The researchers opt for Cox regression hazard models to analyse the time-to-event relationship from baseline predictors to the onset of cognitive issues. Sensitivity analyses reinforce the robustness of their findings, revealing that loneliness is a significant predictor of cognitive impairment, even after adjusting for variables such as age, sex, education, and depressive symptoms.

    Lee et al. (2020) focus on exploring loneliness among older adults in the Czech Republic. They employ the UCLA-L scale to measure loneliness and use the EURO-D scale to evaluate mental and emotional health. While the study aims to understand the relationship between mental and physical health, its methodology is limited to regression analysis, analysis of variance (ANOVA), and descriptive statistics, without addressing the aforementioned endogeneity issues.

    Hajek and König (2022) employ SHARE longitudinal data and utilize linear fixed-effects regression to account for unobservable variables while investigating the factors associated with loneliness in older Europeans. Their analysis reveals that loneliness intensifies with factors such as aging, alterations in marital status, reductions in log income, deteriorating self-assessed health, and functional decline. Interestingly, they found no correlation between changes in chronic diseases and shifts in loneliness levels.

    In the second category of research papers on loneliness, the following studies were chosen for their use of pandemic-related data. Starting with a paper by Santini et al. (2021-A)
    The second study, by Atzendorf et al. (2022), examines the mental well-being of retired adults in various European countries during the COVID-19 pandemic, with a specific focus on loneliness and depression. The researchers utilized the SHARE Corona Survey, supplemented by the Oxford Government Response Tracker (OxGRT), to gather data on individual feelings of loneliness and depression with respect to pre-pandemic times, and on the stringency of epidemic control measures. Their methodological approach involved multilevel binary logistic regression models that incorporated both individual and country-level variables. The authors find significant differences between countries in the prevalence of increased feelings of depression and loneliness, particularly for the oldest in the sample. Specifically, the number of deaths explains 32.4\% of the country variance in depression and 20.7\% in loneliness.
    The third study, conducted by Arpino et al. (2022), assesses the effects of the COVID-19 pandemic on loneliness in older adults. It specifically explores how variables such as childlessness and lack of a partner contribute to feelings of loneliness. The researchers chose to use the most recent wave of the SHARE dataset for their analysis. Employing a logistic model, they focused on the binary outcome variable of 'loneliness.' Their findings reveal that 11.6\% of respondents felt lonelier during the pandemic, while the overall prevalence of depression rose by 0.8\%. Being childless or unpartnered was a significant risk factor for increased feelings of loneliness. 

    Finally, a paper by Santini et al. (2020) stood out by addressing endogeneity in the analysis of the relationship between social disconnectedness, perceived isolation, and symptoms of depression and anxiety in older adults using longitudinal data from the National Social Life, Health, and Aging Project (NSHAP) in the USA. 
    The method of choice is a random intercept cross-lag panel model with maximum likelihood estimation. According to the authors, this approach aims to establish whether the associations might have been obtained spuriously based on stable third variable traits that were not controlled for. The authors also acknowledge the potential for measurement error, noting that results could vary if mental health were assessed through clinical evaluations rather than screening tools. Additionally, they recognize unaccounted-for confounders like stressful life events or a family history of mental disorders. 
    Their findings indicate that social disconnectedness leads to perceived isolation, which subsequently predicts depression and anxiety. To address concerns of reverse causality, the authors also explored the reverse relationships between variables and found evidence supporting bi-directional influences.


\subsection{Mental Health and Social Capital}
    Mental health and social capital are linked by a mutually reinforcing relationship, creating a cycle that affects both individual and collective well-being. Poor mental health can hinder an individual's ability to accumulate social capital by reducing productivity and limiting social engagement. Conversely, a lack of social capital can exacerbate mental health issues due to diminished social support, community cohesion, and access to quality information. Additionally, societal stigma and economic disadvantages associated with low social capital further impact mental well-being.

    %   sirven debrand  2012
    In their 2012 study, Sirven and Debrand employ panel data from the SHARE and SHARELIFE surveys to explore the causal relationships between social capital and health in an older European population. The authors employ a comprehensive set of baseline health indicators, encompassing both physical dimensions (such as poor self-rated health, limitations in Activities of Daily Living (ADL), General Activity Limitations Indicator (GALI), mobility restrictions, and low grip strength) and mental aspects (such as depressive symptoms and cognitive impairment). Given the bidirectional causality between mental health and overall well-being, addressing endogeneity is crucial for establishing the causal implications of their findings. While acknowledging the merits of an Instrumental Variables (IV) approach, the authors express reservations about its ability to accurately assess the impact of social capital on health when using various social capital measures. Instead, they opt for a bivariate recursive probit model, incorporating lagged values of the dependent variables to account for endogeneity.
    The results suggest a reciprocal causal relationship between social capital and health. Specifically, past health statust influences individual health in the following period, and the same relationship holds for social participation. The impact of health on social capital is found to be more substantial than the reverse, indicating that health may serve as a more potent driver for the accumulation or depletion of social capital. 

    %   murayama        2013
    Murayama et al. (2013) investigates the longitudinal effects of bonding and bridging social capital on self-rated health, depressive mood, and cognitive decline among older Japanese individuals. Utilizing panel data from the Hatoyama Cohort Study, the research focuses on social capital as the key independent variable, where bonding social capital is assessed based on the individual's perception of neighborhood and network homogeneity, while bridging social capital is assessed based on the individual's perception of network heterogeneity.
    The study finds that stronger perceived neighborhood homogeneity is inversely associated with poor self-rated health and depressive mood. However, neither bonding nor bridging social capital was significantly associated with cognitive decline.
    The authors employ logistic regression models to carry out their analysis, however, they do not explicitly address the critical issue of endogeneity, particularly the problem of reverse causality between social capital and health outcomes. 

    %   eshan de silva  2015
    Ehsan and De Silva (2015) present a systematic review that investigates the association between social capital and common mental disorders, such as depression, anxiety, and PTSD, using validated measurement tools. While the review includes a large number of studies, it does not directly tackle the issue of endogeneity in the methodologies of the reviewed works. Additionally, the authors note that the majority of the studies are situated in middle to high-income countries, limiting the generalizability of the findings to lower-income settings.

    %   riumallo        2014
    A paper focused on yet another high-income country is Riumallo et al. (2014), which explores the relation between social capital and both self-rated health and biological health in Chile using data from the Chilean National Health Survey (2009-2010). Using an IV approach with a variety of instruments, they define the dependent variable using a binary indicaotr of self-rated health, depression, hypertension or diabetes. Social capital is captured by a questionnaire inquiring about social support, generalized trust and neighborhood trust. The study uses recent crime victimization and aggregate social capital as instruments, and finds that all social capital indicators have an association with depression. Some social capital indicators are associated with self-rated health, hypertension and diabetes above age 45.

    %   landestedt      2016
    A paper by Landstedt et al. (2016) focuses on the longitudinal relationship between individual-level structural social capital, measured as civic engagement, and depressive symptoms from age 16 to 42 in Swedish men and women, using data from the Northern Swedish Cohort. Civic engagement is measured by a single-item question reflecting the level of engagement in clubs or organizations, and depressive symptoms are measured with an index.
    Methodologically, the authors employ cross-lagged structural equation models separated by gender in order to analyze the directions of associations between civic engagement and depressive symptoms. The directionality between social capital and mental health is established with the use of a cross-lagged structural equation model, which explains present values of the dependent with past values of the independent variable. Results show that both civic engagement and depressive symptoms are stable across time, with male civic engagement being inversely related to depressive symptoms in adulthood, while no such relationship is observed for women. 

    %   cohen cline     2018
    Finally, a paper by Cohen-Cline et al. (2018) explores the relationship between social capital and depression, utilizing a sample of same-sex twin pairs. Symptoms of depression are measured with the Patient Health Questionnaire (PHQ-2); social capital is conceptualized into cognitive and structural domains, with the former including sense of belonging, neighborhood social cohesion, trust and workplace connections, and the latter is measured through volunteerism, community participation, and social interaction.
    The twin design combined with a Poisson model offer an approach that controls for genetic and environmental confounders. However, this does not necessarily translate to solving the ever-present issue of reverse causality. 
    Results show that all measures of cognitive social capital and neighborhood characteristics are associated with less depressive symptoms. 
    

\subsection{Mental Health and Social Networks}
    The topic of social capital is tightly linked with the nature of social networks since their quality serves as a critical dimension of social capital, shaping its effectiveness and impact on mental well-being. High-quality networks, characterized by strong, trust-based relationships, not only facilitate the exchange of valuable information but also provide emotional support and a sense of belonging. These factors contribute to a more robust form of social capital, which in turn positively influences mental health by building resilience and fostering self-esteem (WHO, 2022). Conversely, low-quality social networks, marked by weak ties and low levels of trust, can diminish social capital and exacerbate mental health issues. Therefore, the quality of one's social network is a pivotal factor in the symbiotic relationship between social capital and mental health.
    An individual's social network can be characterized by three key dimensions: the quantity of connections, the quality of those connections, and the geographical proximity to other network members. 
        

    %   shiovitz    2010
    Shiovitz-Ezra and Leitsch (2010) explores the associations between objective and subjective social network characteristics and their impact on loneliness in older adults. Mental health is measured with the R-UCLA scale, and subjective measures such as eyesight and hearing loss. The paper distinguishes between objective indicators like frequency of contact with social network members and subjective perceptions of social ties, such as the quality of marriage or familial relationships. Results show that a larger portion of the variance in loneliness perception is explained in the non-married sub-sample, 13\%, relative to the married or cohabiting sub-sample, 7\%.
    The empirical strategy of the paper involves the use of hierarchical linear regression models applied to a cross-section of the NSHAP dataset. This specification does not properly account for reverse causality:  individuals who are lonely or have mental health issues might have fewer social interactions or perceive their social networks differently. In addition, the model does not adequately address likely omitted variable bias and simultaneity.

    %   gu          2020
    Another correlative paper is found in Gu (2020), which aims to explore the impact of neighborhood social networks on the mental well-being of women residents in a middle-class urban neighborhood in Seoul, South Korea. The study employs a phenomenological qualitative approach, an approach that translates to dialogical interviews to understand the pehonmenon, and focuses on 18 full-time or part-time housewives with children. Results are ambiguous and highly context specific, showing both positive and negative effects on the women's well-being.

    %   santini     2021-B
    Santini et al. (2021-B) investigates the moderating role of social network size in the relationship between formal social participation and mental health outcomes among older adults in Europe, focusing specifically on quality of life and symptoms of depression. The dataset of choice is SHARE (waves from 2011 and 2013) to investigate formal social participation and social network size, with their impact on quality of life and depressive symptoms. The moderating role of formal social participation is investigated through a linear regression model with two possible outcomes: quality of life, as measured by the CASP-12 scale, and depressive symptoms, measured with the EURO-D scale. Although results show that individuals with few social ties may benefit from social participation via a reduction in depressive symptoms and an increase in quality of life, the specification choices raise concerns analogous to those discussed for Shiovitz-Ezra and Leitsch (2010).

    %   coleman     2022
    Finally, Coleman et al. (2022) examine how social networks prior to the pandemic influenced older adults in perceived risk of COVID-19, preventative behavior and mental health outcomes such as loneliness, depression and enxiety. The authors distinguish between bridging and bonding social capital. Specifically, bridging social capital refers to the benefits derived from a vast and diverse social network, while bonding social capital refers to the benefits derived from strong and close ties in a social network. The former manifests in weak ties, and is found to predict a higher perceived risk of COVID-19, as well as more preventative behaviors; the latter is associated with less perceived COVID-19 risk, fewer precaution, but better mental health outcomes. The authord fit 60 models using ordinary least squares (OLS) regression for for continuous outcome variables (risk perception, loneliness, stress), binomial regression for count outcomes (health precautions, depression, anxiety), and generalized linear models (GLM). To accredit causal interpretation of the results, controls for baseline mental health are included in the model. According to the authors, the use of a cross-lagged approach and the timing of data collection should mitigate concerns of reverse causation.
    Results show that mean density of the network, mean tie strength and strength of the weakest tie are significantly associated with loneliness. For depressive symptoms, lower values are associated with mean support functions received from the network, mean tie strength, and strength of the weakest tie. Finally, proportion of frequent contact, diversity, and strength of weakest tie are associated with lower anxiety, while network density is associated with higher anxiety, therefore suggesting that bonding capital can be negatively associated with mental health. 



\subsection{GAPS IN THE LITERATURE: Comparative Analysis}


\subsection{Conclusion}



%The overarching question guiding this study is: "How does mental health, as measured by various indicators, affect individual outcomes such as labor market participation, community activity, and social network strength?" To answer this question, we employ a dataset known as the Survey of Health, Ageing and Retirement in Europe (SHARE), which offers a rich array of variables pertinent to our study.



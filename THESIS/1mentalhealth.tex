\chapter{Mental Health}

% frame the chapter
Increasingly recognized as a crucial factor for well-being, mental health carries significant economic implications that are often overlooked in favor of more easily quantified conditions, such as physical health. Nevertheless, recent events such as the COVID-19 pandemic shed light on the importance of psychological welfare. 

% economic relevance (social capital, productivity)
Mental health is an economically relevant phenomenon with far-reaching implications that extend beyond individual well-being. Poor mental health often leads to reduced productivity, increased absenteeism, and higher turnover rates in the workplace, directly impacting an organization's bottom line (OECD/EU (2018), OECD/EU (2022)). Furthermore, it places a significant burden on healthcare systems through increased medical costs and utilization of services. The indirect costs, such as loss of income due to disability and the ripple effects on families and communities, further amplify its economic relevance (OECD/EU (2022)). Therefore, investing in mental health not only enhances individual quality of life but also has the potential for significant economic returns, framing it as a key opportunity in the context of social capital accumulation.

This chapter aims to shed light on the definitions, statistics and dynamics of the topic, with the aim of providing the reader with comprehensive and up to date knowlege in this realm. 

\section{Defining mental health}
% what is MH in general 
Mental health can be defined as a state of psychological well-being which allows people to cope with demands of life, realize their abilities, learn and work well while contributing to their community. It represents a crucial feature of personal and collective socio-economic develpoment, involving psychological, emotional and social welfare, and affecting how people think, feel and act. 

Being mentally healthy goes beyond the mere absence of clinically relevant mental conditions, it encompasses self-esteem, resilience, relationships. 
Conditions that affect mental health include mental disorders, psychosocial disabilities and mental states associated with impaired functioning, or risk of self-harm. Those affected by these conditions are more likely to report lower mental well-being. 

\section{Exploring the data}
% define relevant MH issues --> depression, anxiety


% statistics using WHO and EUROSTAT data references
    % risk factors and general trends
\paragraph{Gender differences}
\paragraph{Cohort specificity}

\section{Measurement via psychometric tools}
% how is it measured






\section{MH and COVID19}
